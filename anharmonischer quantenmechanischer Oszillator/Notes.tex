
\documentclass[a4paper,11pt,DIV=12,oneside]{scrreprt}

% input encoding
\usepackage[utf8]{inputenc} 
% font encoding            
\usepackage[T1]{fontenc}
\usepackage{lmodern}  	
% language		    
\usepackage[ngerman]{babel}              

% paragraph indent
\setlength{\parindent}{0pt}
% captions
\setkomafont{captionlabel}{\sffamily\bfseries}
%\addtokomafont{caption}{\centering} 
\setcapindent{0pt}
% figures
\usepackage{graphicx}
\graphicspath{{figures/}}
% positioning of figures and tabels
\usepackage{float}     
% extentions for tabular environment                          
\usepackage{multirow}
% mathematical formatting                         
\usepackage{amsmath,amsfonts}

% colors
\usepackage{xcolor}
% TikZ
\usepackage{tikz}
\usetikzlibrary{arrows,patterns}
\tikzset{>=angle 45}


\renewcommand{\vec}{\boldsymbol}

\begin{document}



\setlength{\tabcolsep}{12.5pt}     
\renewcommand{\arraystretch}{1.25}

\begin{gather}
\left(-\frac{\hbar^2}{2m}\Delta + \frac{1}{2}m\omega^2x^2\right)\psi(x) = E\,\psi(x). 
\end{gather}

\begin{gather}
\xi = \sqrt{\frac{m\omega}{\hbar}}\,x \quad\text{und}\quad n = \frac{2E}{\hbar\omega}
\end{gather}

 \begin{gather}
\psi(\xi) = H(\xi)\,\mathrm e^{-\xi^2/2} 
\end{gather}

  
\begin{gather}
H''(x) - 2x H'(x) + 2\nu H(x) = 0 \quad\text{mit}\quad \nu = 0,1,2,\dots 
\end{gather}


    
\begin{gather}
H_\nu(x)=(-1)^\nu e^{x^2}\frac{d^\nu}{dx^\nu}\mathrm e^{-x^2}
\end{gather}

    
\begin{table}[H]
\centering
\captionabove{Hermit-Polynome für $H_\nu(x)$ für $\nu = 0$ bis $3$.}
\begin{tabular}{cl} \hline
$\nu$ & $H_\nu(x)$ \\ \hline	
0 & $1$ \\
1 & $2x$ \\
2 & $4x^2 - 2$ \\
3 & $8x^3 - 12x$ \\ \hline
\end{tabular}
\label{Tab:Sphärische-Besselfunktionen}
\end{table}

\begin{gather}
\psi_\nu(x)= \left(\frac{m\omega}{\pi\hbar}\right)^\frac{1}{4}\frac{1}{\sqrt{2^\nu\nu!}} \,H_\nu\left(\sqrt{\frac{m\omega}{\hbar}}x\right)\,\mathrm e^{-\frac{1}{2}\frac{m\omega}{\hbar}x^2}.
\end{gather}

\begin{figure}[H]
\centering
\begin{tikzpicture}[yscale=1.35,xscale=1.0,semithick,samples=250]
% Achsen
\draw[->] (-4,0) -- (4.5,0) node[right] {$x$};
\draw[->] (0,0) node[below] {0} -- (0,5) node[above] {$E$};
% Energie
\foreach \n/\color in {0/red,1/blue,2/green,3/magenta} {
   \draw[dashed,thin] (-3.75,\n+0.5) node[left,color=\color] {$E_\n$} -- (3.75,\n+0.5);
   \node[color=\color] at ({1.15+sqrt(2*\n+1)},\n+0.85) {$\psi_\n(x)$};
}
% Potential
\draw[domain=-2.85:2.85,samples=100,thick] plot(\x,{0.5*\x*\x});
\node at (-2.85,4.5) {$V(x) = \dfrac{1}{2}m\omega^2x^2$};
% Wellenfunktionen
\draw[domain=-3.75:3.75,red,thick] plot(\x,{0.5*exp(-0.5*\x*\x) + 0.5});
\draw[domain=-3.75:3.75,blue,thick] plot(\x,{0.5*1/sqrt(2)*\x*exp(-0.5*\x*\x) + 1.5});
\draw[domain=-3.75:3.75,green,thick] plot(\x,{0.5*1/sqrt(8)*(4*\x*\x-2)*exp(-0.5*\x*\x) + 2.5});
\draw[domain=-3.75:3.75,magenta,thick] plot(\x,{0.5*1/sqrt(48)*(8*\x*\x*\x-12*\x)*exp(-0.5*\x*\x) + 3.5});
\end{tikzpicture}
\end{figure}

\end{document}
